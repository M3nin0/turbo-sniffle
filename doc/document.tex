\documentclass[
	12pt,				% tamanho da fonte
	%openright,			% capítulos começam em pág ímpar (insere página vazia caso preciso)
	%twoside,			% para impressão em recto e verso. Oposto a oneside
	openany,			%Para nao pular folhas quando um paragrafo novo começa. Oposto de Twoside e openright
	a4paper,			% tamanho do papel.
	chapter=TITLE,		% títulos de capítulos convertidos em letras maiúsculas
	section=TITLE,		% títulos de seções convertidos em letras maiúsculas
	%subsection=TITLE,	% títulos de subseções convertidos em letras maiúsculas
	%subsubsection=TITLE,% títulos de subsubseções convertidos em letras maiúsculas
	english,
	brazil				% o último idioma é o principal do documento
]{abntex2}
\usepackage[brazil]{babel}
\usepackage[utf8]{inputenc} %Pacote de linguas
\usepackage[normalem]{ulem}
\usepackage[T1]{fontenc}
\usepackage{lipsum}
\usepackage{cmap}
\usepackage{graphicx}
\usepackage[brazilian, hyperpageref]{backref}
\usepackage[alf]{abntex2cite} % Citações padrão ABNT
\usepackage{rotating}
\usepackage{float}
\usepackage{color}
\usepackage{listings}    
\usepackage{inconsolata}
%\usepackage[breaklinks=true]{hyperref} %Pacote para correção de problemas com sumário

\usepackage{listings}

\lstset{language=sh}

\newcommand{\colcheted}{] }
\newcommand{\colchetee}{[}

% Params
% 1 - Escala da figura
% 2 - Caminho absoluto para a figura
% 3 - Legenda da figura
\newcommand{\includeImage}[3] {

\begin{figure}[H]
 	 \centering
  		\includegraphics[scale=#1]{#2}
  	\caption{#3}
\end{figure}

}

\title{Sistemas operacionais II \\ Trabalho 1 - Cluster de máquinas}
\date{\today}
\autor{Akira Kotsugai \\ Felipe Menino Carlos \\ Weslei Luiz}

\setlength{\parindent}{1.3cm}
\frenchspacing

% Adicionando idioma
\selectlanguage{brazil}

\begin{document}
\maketitle

\newpage
\tableofcontents
\newpage
\listoffigures
\newpage
\listoftables

\chapter{Contextualização}

Linux é um termo utilizado para se referir a sistemas operacionais que utilizem o núcleo Linux. O núcleo ou kernel Linux foi desenvolvido pelo programador finlandês Linus Torvalds, inspirado no sistema Minix. O seu código fonte está disponível sob a licença GPL (versão 2), para que qualquer pessoa o possa utilizar, estudar, modificar e distribuir livremente, de acordo com os termos da licença. Atualmente este sistema operacional é muito usado em servidores (Web, E-mail, Banco de Dados...), e também como ferramenta administrativa para segurança em redes de computadores. Saber instalar e configurar este sistema operacional é importante, e uma falha pode causar um resultado catastrófico.

O objetivo deste trabalho é realizar uma configuração de cluster, com duas máquinas no mínimo, instaladas e configuradas de acordo com os seguintes requisitos:

\begin{itemize}
	\item Sistema operacional: Debian
	\begin{itemize}
		\item Sem interface gráfica;
		\item Partições separadas para o /home e /var. /home com no máximo 100mb e /var com 3gb. O formato das partições será o EXT3
	\end{itemize}

	\item As máquinas deverão estar na mesma rede. Mesma máscara de rede e faixa de IP.

	\item A comunicação entre elas deverá ser habilitada por ssh, e não deve ser permitido a uma máquina realizar conexão remota com outra que não pertença ao cluster, exceto o gateway. O acesso ao cluster por máquinas externas deverá ser habilitado, e por isso o gateway deverá ter duas interfaces de rede, uma para comunicação interna e outra para comunicação externa.

	\item Deverá existir uma máquina gateway, ela irá fornecer acesso as outras máquinas, à Internet e a conexão remota externa, ou seja, alguém poderá realizar ssh para o gateway, e a partir daí acessar as máquinas do cluster.

	\item Não será permitido ssh como root direto. E o usuário administrador não deverá ter acesso a senha do usuário root.

	\item Os usuários do cluster deverão ter contas em cada máquina, e serão pelo menos 3 usuários. Deve existir um usuário administrador, responsável por gerenciar os demais. Este administrador será o único com acesso a poderes de root em todas as máquinas. Cada usuário deverá ter uma quota em disco de no máximo 50mb, para isso será necessário estudar o funcionamento do pacote quota.

	\item Os sistemas deverão ter os seguintes grupos:
		\begin{itemize}
			\item Arquivadores: Usuários responsáveis pelo gerenciamento de arquivos
			\item Agendadores: Usuários responsáveis pelo agendamento de tarefas
		\end{itemize}		    	
    	
	\item O usuário administrador deverá distribuir os demais nos grupos.

	\item Para cada grupo deverá ser criado uma pasta no /var. O acesso deverá ser restrito ao grupo, ou seja, usuários que não sejam dos grupos supracitados não poderão acessar o conteúdo das pastas.

\end{itemize}

\chapter{Criação do cluster}

Neste capítulo será descrito as etapas tomadas para a criação do cluster. 

\section{Arquitetura}

A arquitetura adotada para a solução dos problemas apresentados, seguirá o modelo cliente/servidor, e pode ser visualizada abaixo:

\includeImage{0.7}{imgs/others/topologia.png}{Topologia do projeto}

Nas próximas seções serão apresentados os passos para a configuração desta arquitetura. É importante lembrar que, os passos estão na mesma sequência em que as configurações foram realizadas.

\section{Instalação do sistema operacional}

O primeiro passo para a configuração do \textbf{cluster} será a instalação do sistema operacional. Nesta etapa foi realizada a divisão das partições, para a utilização separada dos diretórios \textbf{/home}, com até 100 MB de espaço e o \textbf{/var} com até 3GB de espaço livre.

Veja abaixo os passos da instalação.

\includeImage{0.5}{imgs/1_instalacao/1.png}{Tela inicial de instalação}

Na imagem que segue, é realizada a configuração das partições, essas foram configuradas utilizando o \textbf{EXT3}, para que em um passo futuro a configuração do pacote \textbf{quotes}, seja realizada sem problemas.
\includeImage{0.5}{imgs/1_instalacao/2.png}{Configuração das partições}

O sistema instalado tem apenas os serviços básicos
\includeImage{0.5}{imgs/1_instalacao/3.png}{Definições dos serviços/\textit{softwares} padrão}

A etapa abaixo, demonstra as partições criadas anteriormente.
\includeImage{0.5}{imgs/1_instalacao/4.png}{Confirmação da separação das partições}

Após realizar os passos demonstrados acima, a instalação do sistema operacional foi realizada.

\section{Configuração das interfaces de rede}

Nesta etapa será realizado as interfaces de rede, no \textit{gateway} e no \textit{host}.

\subsection{Configuração do gateway}

No caso do \textit{gateway}, ele terá duas interfaces de rede, uma para realizar a comunicação com a rede externa (\textit{internet}), e outra para a comunicação interna, entre as máquinas do cluster.

As interfaces do gateway são:
\begin{itemize}
	\item \textbf{enp0s3} - Rede externa
		\begin{itemize}
			\item IP: Dinâmico
		\end{itemize}
	\item \textbf{enp0s8} - Rede interna
		\begin{itemize}
			\item IP: 10.20.30.1
			\item Rede: 255.255.255.0 (/24)
		\end{itemize}
\end{itemize}

Abaixo é demonstrado o arquivo de configuração da interface de rede.

\includeImage{0.5}{imgs/2_configuracao_rede/gateway/1.png}{Configuração de rede - Gateway}

O arquivo representado na imagem é o \textbf{/etc/network/interfaces}

\subsection{Configuração do host}

Diferente do \textit{gateway}, os \textit{hosts} terão apenas uma interface, que será conectada com o \textit{gateway}.

A configuração seguida na interface dos hosts foi a seguinte:
\begin{itemize}
	\item Host 1
		\begin{itemize}
			\item IP: 10.20.30.2
			\item Rede: 255.255.255.0 (/24)
		\end{itemize}
	\item Host 2
		\begin{itemize}
			\item IP: 10.20.30.3
			\item Rede: 255.255.255.0 (/24)
		\end{itemize}
\end{itemize}

\includeImage{0.5}{imgs/2_configuracao_rede/host/1_host_1.png}{Configuração de rede - Host 1}
\includeImage{0.5}{imgs/2_configuracao_rede/host/1_host_2.png}{Configuração de rede - Host 2}

% Nesta parte será adicionado as configurações de usuários e grupos
\section{Gerenciamento dos usuários e grupos}

Neste capítulo, todo o gerenciamento de grupos e usuários é abordado. Três usuários foram adicionados para atender os requisitos citados na contextualização:

%Lista ordenada
\begin{itemize}
	\item Usuário administrador com privilégios de \textbf{root} para gerenciar todos os usuários, sem possuir a senha do root;
	\item Administrador para o grupo Arquivadores;
	\item Administrador para o grupo Agendadores;
\end{itemize}

\subsection {Adicionar usuários ao sistema}
O comando utilizado para adicionar os usuários "admin", "agendador" e "arquivador", foi o \textbf{adduser} conforme exemplo abaixo:

\includeImage{0.7}{imgs/5_gerenciamento_grupos/adduser_exemplo.png}{Adicionando usuários com adduser}


\subsection{Configuração usuário admin}
Para conceder privilégios de \textbf{root} ao usuário \textbf{admin} foi utilizado o pacote \textbf{sudo}, este pacote eleva a permissão de usuários comuns, para que possam executar tarefas de administradores quando necessário, digitando \textbf{sudo}, antes de comandos que são autorizados apenas para o root. Para realizar a instalação do pacote use:

\begin{lstlisting}
# apt-get install sudo
\end{lstlisting}

\includeImage{0.7}{imgs/5_gerenciamento_grupos/instalando_sudo.png}{Instalação do pacote sudo}

As permissões de administrador para usuários comuns, são configuradas no arquivo \textbf{sudoers} localizado em \textbf{/etc/sudoers}, preferencialmente utilizando o comando \textbf{visudo}, os parâmetros definidos são:

\begin{itemize}
	\item Máquinas em que os comandos poderão ser executados;
	\item Usuários que poderão executar os comandos;
	\item Comandos permitidos ou não permitidos.
\end{itemize}

Na imagem abaixo as permissões do usuário admin são configuradas:

\includeImage{0.7}{imgs/5_gerenciamento_grupos/adicionando_permissoes_de_root_para_usuario_admin.png}{Configuração do arquivo sudoers}

Após esta configuração o usuário admin conseguirá executar comandos com elevação root, o ponto de exclamação seguido do último ALL, significa que o comando a seguir não poderá ser executado, no caso o \textbf{su} e o \textbf{passwd}. Se estes comandos estivessem habilitados, seria possível obter acesso root ou alterar a senha do root e conseguir acesso.

\subsection{Criação e configuração dos diretórios}

Os diretórios e grupos \textbf{arquivadores} e \textbf{agendadores} foram criados para atender os dois últimos requisitos deste capítulo:

\includeImage{0.7}{imgs/5_gerenciamento_grupos/diretorios_grupos_arquivadores_agendadores.png}{Grupos e Diretórios para os agendadores e arquivadores}

Os diretórios \textbf{arquivadores} e \textbf{agendadores} foram alterados de grupo, para seus respectivos administradores com o comando \textbf{chgrp agendadores agendadores\textbackslash}, e \textbf{chgrp arquivadores arquivadores\textbackslash}, seguindo esta sintaxe chgrp \colchetee grupo\colcheted \colchetee diretório\colcheted conforme imagem abaixo:

\includeImage{0.7}{imgs/5_gerenciamento_grupos/alteracao_de_grupo_dos_diretorios.png}{Alteração de grupo dos diretórios agendadores - arquivadores}

Todas as permissões de execução, leitura e escrita foram removidas dos outros e do dono utilizando o comando \textbf{chmod}, para que apenas usuários que pertencerem aos grupos agendadores e/ou arquivadores possam realizar operações nos diretórios:

\includeImage{0.7}{imgs/5_gerenciamento_grupos/permissao_restringida_apenas_para_os_grupos.png}{Permissões - diretórios agendadores e arquivadores}

\begin{table}[h]
\caption{Permissões}
\centering
\begin{tabular}{r|lr}
\cline{1-3}
\multicolumn{1}{|l|}{Usuário} & \multicolumn{1}{l|}{0} & \multicolumn{1}{l|}{- - -} \\ 
\cline{1-3}
\multicolumn{1}{|l|}{Grupo} & \multicolumn{1}{l|}{7} & \multicolumn{1}{l|}{rwx} \\ 
\cline{1-3}
\multicolumn{1}{|l|}{Outros} & \multicolumn{1}{l|}{0} & \multicolumn{1}{l|}{- - -} \\ 
\cline{1-3}
\end{tabular}
\end{table}

\subsection{Configuração dos grupos}

Os usuários agendador e arquivador, foram adiciondos em seus grupos agendadores e arquivadores, respectivamente:

\includeImage{0.7}{imgs/5_gerenciamento_grupos/adicionando_usuarios_aos_grupos.png}{Incluindo usuários a seus grupos}

Os mesmos foram definidos como administradores de seu grupo para que realizem o gerenciamento de usuários com o comando \textbf{gpasswd}:

\includeImage{0.7}{imgs/5_gerenciamento_grupos/definindo_administrador_dos_grupos.png}{Definição de usuários administradores}

Desta forma toda a administração dos grupos pode ser feita por usuários sem elevação de root.

\section{Configuração do quota}

A \textbf{quota} é uma ferramenta que facilita o gerenciamento de espaços, e limite para grupos e usuários. No tópico de instalação do sistema, foi mencionado que o particionamento seria criado utilizando o \textbf{EXT3}, isto foi feito por conta do quota, é importante citar este tópico pois, este é um pré-requisito para a utilização do pacote. Veja abaixo os passos utilizados na configuração do quota. 

Instalação do pacote
\begin{lstlisting}
# apt install quota
\end{lstlisting}

Após realizar a instalação, será necessário definir quais partições farão a utilização do \textbf{quota}, para isso é feito o acesso a \textbf{/etc/fstab}, dentro deste arquivo, é inserido nas opções da partição escolhida a opção \textbf{usrquota}, isso porque neste caso será feito o controle através de usuários. Aqui o quote será aplicado em todas as partições, para que o usuário seja limitado ao máximo no uso do disco.

\includeImage{0.7}{imgs/3_configuracao_quota/1.png}{Arquivo de configuração de quota}

As configurações de \textbf{quota} demonstradas acima, estão replicadas em todas as máquinas \textit{host} do cluster. 

Após realizar as configurações acima, será necessário reinicializar o sistema. Antes de continuar as configurações do \textbf{quotes}, veja a explicação de alguns parâmetros que serão utilizados:
\begin{itemize}
	\item \textbf{-a} - Checar todos os sistemas de arquivos em \textbf{/etc/fstab} que estão habilitados como 'automount';
	\item \textbf{-u} - Checa \textbf{quotas} de usuários (Esta é uma opção padrão, ou seja, mesmo quando não especificada, será utilizada);
	\item \textbf{-g} - Checa \textbf{quotas} de grupos;
	\item \textbf{-v} - Mostra mais detalhes na saída do comando.
\end{itemize}

Com o conhecimento sobre cada um dos parâmetros utilizados no \textbf{quotas}, será agora realizado a continuação da configuração

Para a continuação, será necessário parar os serviços de \textbf{quotas}, isto porque no momento da inicialização ele é iniciado.
\begin{lstlisting}
# quotaoff -augv 
\end{lstlisting}

Com o serviço parado faça a verificação das \textbf{quotas} de disco, em todos os sistemas de arquivos que estão em \textbf{/etc/fstab}.
\begin{lstlisting}
# quotacheck -augv
\end{lstlisting}

Neste momento o serviço já está configurado, porém há um passo a ser realizado, adicionar os usuários que terão limites de uso, neste caso, será os usuários comuns, criados na seção de gerênciamento de usuários.

A edição será inicialmente feita apenas para um usuário, neste caso o \textbf{agendador}.
\begin{lstlisting}
# edquota agendador
\end{lstlisting}

Dentro deste arquivo insira os parâmetros como demonstrado abaixo.
\includeImage{0.7}{imgs/3_configuracao_quota/2.png}{Configuração de \textbf{quota} para usuário}

Na figura apresentada anteriormente, há alguns parâmetros que devem ser levados em consideração

\begin{itemize}
	\item \textbf{Filesystem} - Partição que terá a quota do usuário editada. No exemplo \textbf{/dev/sda1} e \textbf{/dev/sda5}
    \item \textbf{blocks} - Número máximo de blocos (especificado em Kbytes) que o usuário possui atualmente;
    \item \textbf{soft} - Restrição mínima de espaço em disco usado. No exemplo 25000 Kbytes (25 MB);
    \item \textbf{hard} - Limite máximo aceitável de uso em disco para o usuário. O sistema de \textbf{quotas} nunca deixará este limite ser ultrapassado. No exemplo 50000 Kbytes (50 MB);
    \item \textbf{inodes} - Número máximo de arquivos (inodes) que o usuário possui atualmente na partição especificada;
    \item \textbf{soft} - Restrição mínima de número de arquivos que o usuário possui no disco;
    \item \textbf{hard} - Restrição máxima de número de arquivos que o usuário.
\end{itemize}

Após inserir a regra para este usuário, será realizada uma cópia dessas configurações para os demais usuário, veja:

\begin{lstlisting}
# edquota -p agendador arquivador
\end{lstlisting}

As regras foram copiadas, agora será necessário fazer as verificações e ativar o serviço

\begin{lstlisting}
# quotacheck -augv
\end{lstlisting}

O comando acima, além de fazer a verificação, cria os arquivos \textbf{aquota.user} e \textbf{aquota.group}, que serão utilizados para criar as regras para usuários e grupos respectivamente.
\begin{lstlisting}
# quotaon -augv
\end{lstlisting}

Com isso as regras de \textbf{quotes} já estarão habilitadas e funcionando.

% SSH e compartilhamento de rede
\section{Configuração dos serviços de rede}

Neste etapa, será demonstrado o processo de configuração dos serviços de rede que irão permitir a comunicação entre as máquinas do cluster. Serão instalados e configurados o \textbf{SSH} e o compartilhamento de rede utilizando o \textbf{IPTABLES}.

\subsection{Configuração do SSH no Gateway}

A configuração do \textbf{ssh} no \textit{gateway} será realizada para que ele aceite conexões na porta \textbf{5678}. O primeiro passo será a instalação do pacote \textbf{ssh}.

\begin{lstlisting}
# apt install ssh
\end{lstlisting}

Após a instalação do pacote, será necessário realizar o acesso ao arquivo de configuração do \textbf{ssh}, este que fica no diretório \textbf{/etc/ssh/sshd\char`_config}, dentro deste arquivo serão alterados os argumentos \textbf{Port} e \textbf{PermitRootLogin}, como demonstrado abaixo:

\includeImage{0.7}{imgs/4_configuracao_servicos_rede/1_gateway_1.png}{Configuração de SSH - Gateway}

Depois de configurado, o acesso não será permitido para o usuário \textbf{root} e a conexão \textbf{ssh} para o \textbf{gateway} só poderá ser realizada na porta \textbf{5678}.

\subsection{Configuração do SSH no Host}

Da mesma forma que demonstrado no tópico anterior, o pacote \textbf{ssh} deverá estar instalado nos hosts.

Com o pacote instalado, será necessário acessar o arquivo \textbf{/etc/ssh/sshd\char`_config}, e modificar apenas a linha de permissão de acesso do \textbf{root}. 

Ao realizar a configuração, a linha a ser alterada ficará como demonstrado abaixo:
\includeImage{0.7}{imgs/4_configuracao_servicos_rede/1_host_1.png}{Configuração de SSH - Host}

As máquinas \textbf{host} terão ainda uma opção para negar o acesso de todos os pedidos de conexão \textbf{ssh} que não venham de máquinas do cluster. Esta configuração será realizada no arquivo \textbf{/etc/hosts.allow}.

Dentro deste arquivo será declarado que qualquer conexão \textbf{ssh} que tenha origem diferente do endereço de rede \textbf{10.20.30.0}, deverá ser recusada. Abaixo há o arquivo já com as definições feitas.

\includeImage{0.5}{imgs/4_configuracao_servicos_rede/1_host_2.png}{Bloqueio de acesso externo ao cluster}


É importante lembrar que todas estas configurações foram feitas nas duas máquinas \textbf{host} presentes no cluster.


\subsection{Configuração do compartilhamento de rede}

Para finalizar o processo de configuração dos serviços de rede, será realizado o compartilhamento dos serviços de \textit{internet} do \textbf{gateway} para os \textbf{hosts}.

O processo de configuração do compartilhamento de rede pode ser visualizado abaixo:

\begin{lstlisting}
# sysctl -w net.ipv4.ip_forward=1
# sysctl -p 
# iptables -X
# iptables -F
# iptables -t nat -X
# iptables -t nat -F 
# iptables -I INPUT -m state --state RELATED,ESTABLISHED -j ACCEPT
# iptables -I FORWARD -m state --state RELATED,ESTABLISHED -j ACCEPT 
# iptables -t nat -I POSTROUTING -o enp0s3 -j MASQUERADE
\end{lstlisting} 

Ao final da execução das linhas demonstradas acima, os \textbf{hosts} terão acesso aos serviços de \textit{internet} fornecidos pelo \textbf{gateway}.
 
% Demonstrar o funcionamento
\section{Testes}

As configurações realizadas nos capítulos anteriores serão evidenciados neste, para demonstrar como o projeto funciona na prática.

\subsection{Partições}

Todas as máquinas particionadas:

\includeImage{0.65}{imgs/testes/1-gateway_particoes.png}{Gateway particionado.}

\includeImage{0.6}{imgs/testes/1-host1_particoes.png}{Host 1 particionado.}

\includeImage{0.6}{imgs/testes/1-host2_particoes.png}{Host 2 particionado.}

\subsection{Configuração de redes das máquinas}

\includeImage{0.6}{imgs/testes/2-gateway_rede_interna_e_externa.png}{Rede externa e interna do Gateway.}

\includeImage{0.6}{imgs/testes/2-host1_rede_interna.png}{Rede interna do Host 1.}

\includeImage{0.6}{imgs/testes/2-host2_rede_interna.png}{Rede interna do Host 2.}

\subsection{Conexões SSH}

\includeImage{0.6}{imgs/testes/3-host1_consegue_dar_ssh_no_gateway.png}{Host 1 dando SSH no Gateway com sucesso.}

\includeImage{0.6}{imgs/testes/3-host1_dando_ssh_no_host2_com_sucesso.png}{Host 1 dando SSH no Host 2 diretamente.}

\includeImage{0.6}{imgs/testes/3-host1_dando_ssh_no_gateway_e_depois_no_host2.png}{Host 1 dando SSH no Host 2 pelo Gateway}

\includeImage{0.6}{imgs/testes/3-dando_ssh_de_outra_maquina_no_gateway.png}{SSH de outra máquina não pertencente ao cluster, no Gateway.}

\includeImage{0.6}{imgs/testes/3-dando_ssh_de_outra_maquina_no_host1_pelo_gateway.png}{SSH de outra máquina não pertencente ao cluster no Host 1, através do Gateway}

\includeImage{0.7}{imgs/testes/3-host2_nao_consegue_dar_ssh_em_maquina_fora_do_cluster.png}{Host 2 tentando dar SSH em máquina fora do cluster sem sucesso.}

\includeImage{0.6}{imgs/testes/5-root_tentando_dar_ssh_sem_sucesso.png}{Root tentando dar SSH sem sucesso.}

\subsection{Poderes de root para o Administrador}

\includeImage{0.6}{imgs/testes/6-admin_deletando_arquivo_do_agendador_com_sucesso.png}{Admin conseguindo apagar arquivo de outro usuário.}

\includeImage{0.6}{imgs/testes/6-arquivador_tentando_apagar_o_arquivo_do_agendador_sem_sucesso.png}{Outro usuário tentando ter poder de root sem sucesso.}

\includeImage{0.55}{imgs/testes/6-admin_nao_consegue_se_tornar_root.png}{Admin não consegue se tornar root.}

\includeImage{0.55}{imgs/testes/6-admin_nao_consegue_mudar_senha_do_root.png}{Admin não consegue mudar senha do root.}

\subsection{Quotas}

\includeImage{0.55}{imgs/testes/6-agendador_baixando_arquivo_de_7mb_com_sucesso.png}{Usuário conseguindo baixar arquivo de 7mb.}

\includeImage{0.55}{imgs/testes/6-agendador_baixando_arquivo_de_50_sem_sucesso.png}{Usuário não conseguindo baixar arquivo de 50mb.}

\subsection{Grupos}

\includeImage{0.6}{imgs/testes/7-pastas_para_os_grupos_com_acesso_restrito.png}{Pastas com acesso restrito para os grupos.}

\includeImage{0.6}{imgs/testes/7-agendador_nao_conseguindo_escrever_na_pasta_de_arquivadores.png}{Agendador tentando acessar pasta de arquivador sem sucesso.}





\chapter{Conclusão}


\end{document}
