\documentclass[
	12pt,				% tamanho da fonte
	%openright,			% capítulos começam em pág ímpar (insere página vazia caso preciso)
	%twoside,			% para impressão em recto e verso. Oposto a oneside
	openany,			%Para nao pular folhas quando um paragrafo novo começa. Oposto de Twoside e openright
	a4paper,			% tamanho do papel.
	chapter=TITLE,		% títulos de capítulos convertidos em letras maiúsculas
	section=TITLE,		% títulos de seções convertidos em letras maiúsculas
	%subsection=TITLE,	% títulos de subseções convertidos em letras maiúsculas
	%subsubsection=TITLE,% títulos de subsubseções convertidos em letras maiúsculas
	english,
	brazil				% o último idioma é o principal do documento
]{abntex2}
\usepackage[brazil]{babel}
\usepackage[utf8]{inputenc} %Pacote de linguas
\usepackage[normalem]{ulem}
\usepackage[T1]{fontenc}
\usepackage{lipsum}
\usepackage{cmap}
\usepackage{graphicx}
\usepackage[brazilian,hyperpageref]{backref}
\usepackage[alf]{abntex2cite} % Citações padrão ABNT
\usepackage{rotating}
\usepackage{float}
\usepackage{color}
\usepackage{listings}    
\usepackage{inconsolata}

\usepackage{listings}

\newcommand{\imagem}[3]{
	\begin{figure}[htb]
		\begin{center}
			\includegraphics[scale=0.5]{#1}
		\end{center}
		\caption{#2}	%\label{#3}
	\end{figure}
}

\title{Sistemas operacionais II \\ Trabalho 1 - Cluster de máquinas}
\date{\today}
\autor{Akira Kotsugai \\ Felipe Menino Carlos \\ Weslei Luiz}

\setlength{\parindent}{1.3cm}
\frenchspacing

% Adicionando idioma
\selectlanguage{brazil}

\begin{document}
\maketitle

\chapter{Contextualização}

Linux é um termo utilizado para se referir a sistemas operacionais que utilizem o núcleo Linux. O núcleo ou kernel Linux foi desenvolvido pelo programador finlandês Linus Torvalds, inspirado no sistema Minix. O seu código fonte está disponível sob a licença GPL (versão 2) para que qualquer pessoa o possa utilizar, estudar, modificar e distribuir livremente de acordo com os termos da licença. Atualmente este sistema operacional é muito usado em servidores (Web, e-mail, Banco de Dados...), e também como ferramenta administrativa para segurança em redes de computadores. Saber instalar e configurar este sistema operacional é importante e uma falha pode causar um resultado catastrófico.

Seu objetivo neste trabalho é entregar uma configuração de cluster, com duas máquinas no mínimo, instaladas e configuradas de acordo com os seguintes requisitos:

- Sistema operacional: Debian
    - Sem interface gráfica.
    - Partições separadas para o /home e /var. /home com no máximo 100mb e /var com 3gb. O formato das partições será o EXT3.

- As máquinas deverão estar na mesma rede. Mesma máscara de rede e faixa de
IP.

- A comunicação entre elas deverá ser habilitada por ssh e não deve ser permitido a uma máquina realizar conexão remota com outra que não pertença ao cluster, exceto o gateway. O acesso ao cluster por máquinas externas deverá ser habilitado, e por isso o gateway deverá ter duas interfaces de redes, uma para comunicação interna e outra para comunicação externa.

- Deverá existir uma máquina gateway e ela irá fornecer acesso, as outras máquinas, à Internet e a conexão remota externa, ou seja, alguém poderá realizar ssh para o gateway e a partir daí acessar as máquinas do cluster.

- Não será permitido ssh como root direto. E o usuário administrador não deverá ter acesso a senha do usuário root.

- Os usuários do cluster deverão ter contas em cada máquina e serão pelo menos 3 usuários. Sendo que deve existir um usuário administrador responsável por gerenciar os demais. Este administrador será o único com acesso a poderes de root em todas as máquinas. Cada usuário deverá ter uma quota em disco de no máximo 50mb, para isso será necessário estudar o funcionamento do pacote quota.

- Os sistemas deverão ter os seguintes grupos:
    - Arquivadores: Usuários responsáveis pelo gerenciamento de arquivos.
    - Agendadores: Usuários responsáveis pelo agendamento de tarefas.

- O usuário administrador deverá distribuir os demais nos grupos.

- Para cada grupo deverá ser criado uma pasta no /var. O acesso deverá ser restrito ao grupo, ou seja, usuários que não sejam dos grupos supracitados não poderão acessar o conteúdo das pastas.

\chapter{Criação do cluster}

\subsection{Instalação do sistema operacional}

\subsection{Configuração das interfaces de rede}

\subsection{Configuração do quota}

\subsection{Gerênciamento dos usuários}

\subsection{Configuração dos serviços de rede}

\subsection{Testes}

\subsection{Conclusão}

\end{document}
