\documentclass[
	12pt,				% tamanho da fonte
	%openright,			% capítulos começam em pág ímpar (insere página vazia caso preciso)
	%twoside,			% para impressão em recto e verso. Oposto a oneside
	openany,			%Para nao pular folhas quando um paragrafo novo começa. Oposto de Twoside e openright
	a4paper,			% tamanho do papel.
	chapter=TITLE,		% títulos de capítulos convertidos em letras maiúsculas
	section=TITLE,		% títulos de seções convertidos em letras maiúsculas
	%subsection=TITLE,	% títulos de subseções convertidos em letras maiúsculas
	%subsubsection=TITLE,% títulos de subsubseções convertidos em letras maiúsculas
	english,
	brazil				% o último idioma é o principal do documento
]{abntex2}
\usepackage[brazil]{babel}
\usepackage[utf8]{inputenc} %Pacote de linguas
\usepackage[normalem]{ulem}
\usepackage[T1]{fontenc}
\usepackage{lipsum}
\usepackage{cmap}
\usepackage{graphicx}
\usepackage[brazilian, hyperpageref]{backref}
\usepackage[alf]{abntex2cite} % Citações padrão ABNT
\usepackage{rotating}
\usepackage{float}
\usepackage{color}
\usepackage{listings}    
\usepackage{inconsolata}

\usepackage{listings}

\lstset{language=sh}

% Params
% 1 - Escala da figura
% 2 - Caminho absoluto para a figura
% 3 - Legenda da figura
\newcommand{\includeImage}[3] {

\begin{figure}[H]
 	 \centering
  		\includegraphics[scale=#1]{#2}
  	\caption{#3}
\end{figure}

}

\title{Sistemas operacionais II \\ Trabalho 1 - Cluster de máquinas}
\date{\today}
\autor{Akira Kotsugai \\ Felipe Menino Carlos \\ Weslei Luiz}

\setlength{\parindent}{1.3cm}
\frenchspacing

% Adicionando idioma
\selectlanguage{brazil}

\begin{document}
\maketitle

\chapter{Contextualização}

Linux é um termo utilizado para se referir a sistemas operacionais que utilizem o núcleo Linux. O núcleo ou kernel Linux foi desenvolvido pelo programador finlandês Linus Torvalds, inspirado no sistema Minix. O seu código fonte está disponível sob a licença GPL (versão 2) para que qualquer pessoa o possa utilizar, estudar, modificar e distribuir livremente de acordo com os termos da licença. Atualmente este sistema operacional é muito usado em servidores (Web, e-mail, Banco de Dados...), e também como ferramenta administrativa para segurança em redes de computadores. Saber instalar e configurar este sistema operacional é importante e uma falha pode causar um resultado catastrófico.

Seu objetivo neste trabalho é entregar uma configuração de cluster, com duas máquinas no mínimo, instaladas e configuradas de acordo com os seguintes requisitos:

\begin{itemize}
	\item Sistema operacional: Debian
	\begin{itemize}
		\item Sem interface gráfica;
		\item Partições separadas para o /home e /var. /home com no máximo 100mb e /var com 3gb. O formato das partições será o EXT3
	\end{itemize}

	\item As máquinas deverão estar na mesma rede. Mesma máscara de rede e faixa de IP.

	\item A comunicação entre elas deverá ser habilitada por ssh e não deve ser permitido a uma máquina realizar conexão remota com outra que não pertença ao cluster, exceto o gateway. O acesso ao cluster por máquinas externas deverá ser habilitado, e por isso o gateway deverá ter duas interfaces de redes, uma para comunicação interna e outra para comunicação externa.

	\item Deverá existir uma máquina gateway e ela irá fornecer acesso, as outras máquinas, à Internet e a conexão remota externa, ou seja, alguém poderá realizar ssh para o gateway e a partir daí acessar as máquinas do cluster.

	\item Não será permitido ssh como root direto. E o usuário administrador não deverá ter acesso a senha do usuário root.

	\item Os usuários do cluster deverão ter contas em cada máquina e serão pelo menos 3 usuários. Sendo que deve existir um usuário administrador responsável por gerenciar os demais. Este administrador será o único com acesso a poderes de root em todas as máquinas. Cada usuário deverá ter uma quota em disco de no máximo 50mb, para isso será necessário estudar o funcionamento do pacote quota.

	\item Os sistemas deverão ter os seguintes grupos:
		\begin{itemize}
			\item Arquivadores: Usuários responsáveis pelo gerenciamento de arquivos
			\item Agendadores: Usuários responsáveis pelo agendamento de tarefas
		\end{itemize}		    	
    	
	\item O usuário administrador deverá distribuir os demais nos grupos.

	\item Para cada grupo deverá ser criado uma pasta no /var. O acesso deverá ser restrito ao grupo, ou seja, usuários que não sejam dos grupos supracitados não poderão acessar o conteúdo das pastas.

\end{itemize}

\chapter{Criação do cluster}

Neste capítulo será descrito as etapas tomadas para a criação do cluster. 

\subsection{Arquitetura}

A arquitetura adotada para a solução dos problemas apresentados, seguirá o modelo cliente/servidor, e pode ser visualizada abaixo:

\includeImage{0.5}{imgs/others/topologia.png}{Topologia do projeto}

Nas próximas seções serão apresentados os passos para a configuração desta arquitetura. É importante lembrar que, os passos estão na mesma sequência em que as configurações foram realizadas.

\subsection{Instalação do sistema operacional}

O primeiro passo para a configuração do \textbf{cluster} será a instalação do sistema operacional. Nesta etapa foi realizada a divisão das partições para a utilização separada dos diretórios \textbf{/home}, com até 100 MB de espaço e o \textbf{/var} com até 3GB de espaço livre.

Veja abaixo os passos da instalação.

\includeImage{0.5}{imgs/1_instalacao/1.png}{Tela inicial de instalação}

Na imagem que segue, é realizada a configuração das partições, essas foram configuradas utilizando o \textbf{EXT3}, para que em um passo futuro a configuração do pacote \textbf{quotes}, seja realizada sem problemas.
\includeImage{0.5}{imgs/1_instalacao/2.png}{Configuração das partições}

O sistema instalado tem apenas os serviços básicos
\includeImage{0.5}{imgs/1_instalacao/3.png}{Definições dos serviços/\textit{softwares} padrão}

A etapa abaixo, demonstra as partições criadas anteriormente.
\includeImage{0.5}{imgs/1_instalacao/4.png}{Confirmação da separação das partições}


Após realizar os passos demonstrados acima, a parte de instalação do sistema operacional foi realizada.

\subsection{Configuração das interfaces de rede}

Nesta etapa será realizado as interfaces de rede, no \textit{gateway} e no \textit{host}.

\subsubsection{Configuração do gateway}

No caso do \textit{gateway}, ele terá duas interfaces de rede, uma para realizar a comunicação com a rede externa (\textit{internet}) e outra para a comunicação interna, entre as máquinas do cluster.

As interfaces do gateway são:
\begin{itemize}
	\item \textbf{enp0s3} - Rede externa
		\begin{itemize}
			\item IP: Dinâmico
		\end{itemize}
	\item \textbf{enp0s8} - Rede interna
		\begin{itemize}
			\item IP: 10.20.30.1
			\item Rede: 255.255.255.0 (/24)
		\end{itemize}
\end{itemize}

Abaixo é demonstrado o arquivo de configuração da interface de rede.

\includeImage{0.5}{imgs/2_configuracao_rede/gateway/1.png}{Configuração de rede - Gateway}

O arquivo representado na imagem é o \textbf{/etc/network/interfaces}

\subsubsection{Configuração do host}

Diferente do \textit{gateway}, os \textit{hosts} terão apenas uma única interface, essa que será conectada com o \textit{gateway}.

A configuração seguida na interface dos hosts foi a seguinte
\begin{itemize}
	\item Host 1
		\begin{itemize}
			\item IP: 10.20.30.2
			\item Rede: 255.255.255.0 (/24)
		\end{itemize}
	\item Host 2
		\begin{itemize}
			\item IP: 10.20.30.3
			\item Rede: 255.255.255.0 (/24)
		\end{itemize}
\end{itemize}

\includeImage{0.5}{imgs/2_configuracao_rede/host/1_host_1.png}{Configuração de rede - Host 1}
\includeImage{0.5}{imgs/2_configuracao_rede/host/1_host_2.png}{Configuração de rede - Host 2}

\subsection{Configuração do quota}

A \textbf{quota} é uma ferramenta que facilita o gerenciamento de espaços e limite para grupos e usuários. No tópico de instalação do sistema, foi mencionado que, o particionamento ser criado utilizando o \textbf{EXT3} foi feito por conta do quota, é importante citar este tópico pois, este é um pré-requisito para a utilização do pacote. Veja abaixo os passos utilizados na configuração do quota. 

1° - Instalação do pacote

\begin{lstlisting}
apt install quota
\end{lstlisting}

Após realizar a instalação, será necessário definir quais partições farão a utilização do \textbf{quota}, para isso acesse \textbf{/etc/fstab}, dentro deste arquivo, insira nas opções da partição escolhida para o \textbf{quote} a opção \textbf{usrquota}, isso porque neste caso será feito o controle através de usuários. Aqui a partição escolhida foi a \textbf{/home}

\includeImage{0.5}{imgs/3_configuracao_quota/1.png}{Arquivo de configuração de quota}

As configurações de \textbf{quota} demonstradas acima, estão replicadas em todas as máquinas \textit{host} do cluster. 

\subsection{Gerênciamento dos usuários}

% SSH e compartilhamento de rede
\subsection{Configuração dos serviços de rede}

% Demonstrar o funcionamento
\section{Testes}

\section{Conclusão}



\end{document}
